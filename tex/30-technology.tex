\section{Технологический раздел}


\subsection{Средства реализации программного комплекса}


\subsubsection{Выбор языка программирования}
Для написания программного обеспечения был использован язык Python 3.10.
Это обусловлено следующими причинами:
\begin{itemize}
	\item имеется опыт разработки на данном языке;
	\item для Python было реализовано множество библиотек, позволяющих работать со сверточными нейронными сетями.
\end{itemize}

\subsubsection{Выбор библиотек для разработки ПО}
Для работы с нейронными сетями была выбрана библиотека tensorflow версии 2.12.

Это обусловлено тем, что на текущий момент tensorflow показывает более высокие показатели производительности, чем другие аналоги, например, pytorch.
Также у автора имеется опыт использования данной библиотеки для разработки сверточных нейронных сетей.

В качестве библиотеки для работы с изображениями был выбран OpenCV, так как в ней присутствуют модули, позволяющие взаимодействовать с функциями из tensorflow.
Для вывода изображений использовалась библиотека matplotlib. 

\subsection{Реализация программного комплекса}

\subsubsection{Нейронная сеть}
Конфигурация всех нейронных сетей основана на методе RIDNET, но количество модулей EAM было уменьшено до трех в целях уменьшения времени, потраченного на обучения моделей.
При этом в качестве функции активации используется сигмоида, что позволило улучшить точность результатов, однако увеличило время обучения сети.

Исходную базу изображений, требуемых для обучения модели, потребовалось разбить на небольшие изображения размером 40 × 40 пикселей, так как это предусмотрено в модели. 
Для увеличения набора тестовых данных некоторые изображения подвергались предварительным преобразованиям, таким как поворот и обрезка.