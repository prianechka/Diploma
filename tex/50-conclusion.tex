\section*{ЗАКЛЮЧЕНИЕ}
\addcontentsline{toc}{section}{ЗАКЛЮЧЕНИЕ}
Была рассмотрена задача удаления импульсных шумов из цветных изображений.
Была описана предметная область: определение понятия шума, причины его возникновения.

Были даны определения основных понятий нейронной сети, описан алгоритм их работы.
Были описаны особенности применения нейронных сетей для борьбы с импульсными шумами в изображениях.

Были рассмотрены и проанализированы существующие методы (медианный фильтр, билатеральный фильтр, DNCNN, RIDNET) для удаления импульсных шумов из изображений, определены их недостатки.

Был разработан метод удаления импульсных шумов из цветных изображений с помощью сверточных нейронных сетей, а также программный комплекс, реализующий графический интерфейс для пользователей.

Были собраны и обработаны данные для обучения моделей, подобрана оптимальная конфигурация.

Был разработан программный комплекс, в который входят следующие модули: модуль разработки обучения моделей, модуль обработки загрязненных изображений.
Приведены результаты обучения модели, основные метрики нейронной сети были показаны графически.
Выбрано оптимальное количество эпох для проведения обучения сети.

Проведено исследование качества созданного программного комплекса.
В результате исследования были выделены преимущества разработанного метода, в том числе: универсальность, независимость от типа импульсного шума, отсутствие предварительной информации о количестве шума, устойчивость результатов в зависимости от процента шума на изображении в сравнении с другими методами.

Таким образом, поставленная цель работы --- разработать метод удаления импульсных шумов из цветных изображений с помощью сверточных нейронных сетей, была достигнута.